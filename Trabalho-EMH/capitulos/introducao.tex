\chapter*[Introdução]{Introdução}
\addcontentsline{toc}{chapter}{Introdução}
\label{chap:Introdução}
A suposição que os mercados seriam informacionalmente eficientes ronda a teoria financeira desde o início do século 20, onde se cogitava que o preço das ações de empresas flutuavam aleatoriamente. A pesquisa sobre se os investidores podem prever com sucesso os preços das ações tem raízes na tese de doutorado de Louis Bachelier, "A Teoria da Especulação" em 1900. Este foi o primeiro esforço a empregar teoria, incluindo técnicas matemáticas e estatísticas para explicar por que o mercado de ações se move da forma como o faz.

Bachelier derivou uma fórmula que explicava o hoje conhecido movimento Browniano, o qual descreve o comportamento de partículas sujeitas a choques aleatórios no espaço.Também desenvolveu o conceito largamente utilizado de processos estocásticos, a análise de movimentos aleatórios entre variáveis estatísticas. Além disso, ele fez a primeira tentativa de precificar instrumentos financeiros como opções e futuros. E ele fez tudo isso em um esforço para explicar porque os preços nos mercados de capitais são impossíveis de prever.

Com o passar do tempo, o movimento browniano passou a ser chamado de passeio aleatório na literatura sobre finanças. Ninguém sabe quem usou essa expressão pela primeira vez, mas se tornou cada vez mais familiar entre os acadêmicos durante a década de 1960. 

A literatura econômica mais recente se inicia com \citeonline{Muth1961} que introduz a teoria de expectativas racionais. Samuelson em 1965 publica um artigo com o título, "Proof that Properly Anticipated Prices Fluctuate Randomly".Em um mercado que apresente eficiência de informação, o preço das ações deve se mover de acordo com um passeio aleatório. Antes do advento das expectativas racionais, os economistas acreditavam que as expectativas eram formadas com informações passadas das variáveis a serem previstas, a esta teoria se chamam expectativas adaptativas.

As expectativas adaptativas foram criticadas pelo fato de as pessoas usarem mais informação do que apenas dados passados em uma única variável para formar suas expectativas desta, elas usam todas as informações relevantes e disponíveis para tanto. Além disso, as pessoas muitas vezes mudam suas expectativas rapidamente à luz de novas informações. Para atender a essas objeções às expectativas adaptativas, John Muth desenvolveu sua teoria de expectativas racionais, que pode ser declarada da seguinte maneira: \emph{as expectativas serão idênticas às previsões ótimas usando todas as informações disponíveis.}

Eugene Fama, conhecido acadêmico e pesquisador do mercado financeiro, tornou-se um dos mais eloquentes propagadores da hipótese de mercados eficientes durante a década de 1960. Sua ideia de um mercado eficiente assim se resume: \emph{"Um mercado no qual os preços sempre refletem completamente todas as informações disponíveis, é chamado de eficiente"}. Esta ficaria conhecida como a forma forte de eficiência nos mercados.

Burton G. Malkiel mais recentemente faz uma definição mais ampla: \emph{"Um mercado de capitais é dito eficiente se este completa e corretamente reflete toda a informação relevante em determinar o preço dos ativos. Formalmente, o mercado é dito eficiente com relação a um conjunto de informação (\ldots) se os preços não forem afetados pela revelação daquela informação para todos os participantes. Ademais, eficiência com relação a um conjunto de informação (\ldots) implica que é impossível obter lucros econômicos ao se operar com base neste conjunto de informação"}.

A principal implicação da hipótese de mercados eficientes - HME, a partir de agora, é que os investidores não podem obter retornos ajustados ao risco além daqueles oferecidos pela carteira de mercado de forma consistente. E por quê não? A competição entre os muitos agentes em busca de lucros elimina qualquer escolha óbvia. A administração ativa de portfólios é um jogo de soma zero mesmo antes dos custos envolvidos, após estes custos e taxas de administração, passa a ser um jogo de soma negativa. De fato, diversas análises empíricas confirmam que a maioria dos administradores tem um desempenho inferior aos índices passivos após as taxas. Essa busca incessante pelo lucro nos mercados faz com que as possibilidades de lucros extraordinários sejam eliminados através de uma espécie de "arbitragem". Os agentes compram aquelas ações que possuem expectativas de lucros maiores e vendem aquelas menos favorecidas. Este movimento faz com que as ações com forte demanda tenham seus preços elevados, e portanto, seus prospectos de retorno futuro diminuídos. O inverso ocorre para as ações de baixa procura. Deste modo, todos os ativos no mercado acabam por oferecer retornos ajustados ao risco equivalentes ao mercado como um todo.

Isso não significa que o mercado esteja sempre correto. De fato, nada na HME ou mesmo na teoria das expectativas racionais, afirma que os agentes não possam estar errados em suas previsões. Como o preço de um ativo financeiro pode ser dado pelo valor presente descontado de todos os fluxos de caixa futuros e ninguém pode prever com precisão esses fluxos futuros e, portanto, os preços de mercado devem estar sempre errados. Os mercados podem ser eficientes, mesmo que às vezes cometer erros notórios na avaliação, como fizeram durante a bolha da Internet. Os mercados podem ser eficientes mesmo se muitos participantes do mercado forem bastante irracionais e se os mercados forem frequentemente fortemente influenciados pela psicologia.

Durante vários anos, o foco principal da pesquisa sobre os mercados de capitais foi determinar se o passeio aleatório é ou não uma descrição válida dos movimentos dos preços de ativos financeiros, como uma forma de testar a hipótese de mercados eficientes.

Mais recentemente com o advento da teoria comportamental nas finanças, o foco passou a ser em buscas de retornos anormais e previsíveis que contraponham-se a HME. Um dos primeiros trabalhos sobre anomalias de retorno a longo prazo é \citeonline{Bondt1985}. Quando as ações são classificadas em com base em retornos passados de três a cinco anos, os vencedores anteriores tendem a ser perdedores futuros e vice-versa. Eles atribuem essas reversões de retorno de longo prazo à reação exagerada do investidor. De Bondt e Thaler argumentam que a reação exagerada à informação passada é uma previsão geral da teoria da decisão comportamental de \citeonline{Kahneman1982}. Assim, pode-se levar a reação exagerada para a previsão de uma alternativa financeira comportamental à eficiência do mercado. Na maior parte, entretanto, a literatura sobre anomalias não dispõe de uma hipótese alternativa a HME.

Se a reação exagerada fosse o resultado geral em estudos de retornos de longo prazo, a eficiência do mercado estaria morta, substituída pela alternativa comportamental. Entretanto, parecem haver tantos estudos que identificam pouca reação dos mercados quanto sobre-reação. Por exemplo, o efeito momento, identificado por \citeonline{Jegadeesh1993} onde as ações com altos retornos durante o último ano tendem a ter retornos elevados nos próximos três a seis meses. Ou seja, é possível obter retornos superiores seguindo uma regra fixa, evidência de sub reação do mercado que proporciona retornos anormalmente altos.


