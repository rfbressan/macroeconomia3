\chapter{As Versões de Eficiência}
\label{cap:fama}

A hipótese de mercados eficientes (HME) vem evoluindo desde a década de 1960, da teoria do passeio aleatório para os preços dos ativos para a hipótese de mercados eficientes em suas formas forte, semi-forte e fraca advogada por Eugene Fama na década de 1970, a quase-eficiência de Grossman e Stiglitz na presença de custos de transação até a crítica comportamentalista mais recentemente.

A HME em sua formulação mais conhecida deve-se a Fama e foi desenvolvida na década de 1960 para então ser popularizada e ganhar adesão acadêmica através de seu influente artigo "Efficient Capital Markets", \citeonline{Fama1970}.

Em sua revisão de 1970, Fama distingue três formas diferentes da
HME.

\begin{enumerate}
	\item A forma “fraca” afirma que toda a informação sobre preços está integralmente refletida nos preços correntes dos ativos, no sentido de que as mudanças de preços atuais não podem ser previstas a partir de preços passados. 
	\item A forma “semi-forte” exige que o preço do ativo mude para refletir integralmente todas as informações publicamente disponíveis e não apenas os preços passados.
	\item A forma “forte” que postula que os preços reflitam totalmente as informações, mesmo que algum investidor ou grupo de investidores tenha acesso exclusivo a alguma informação.
\end{enumerate}

Fama considerava a forma forte de eficiência como \emph{benchmark} para as demais formas de eficiência e na prática, não era alcançável. Entretanto, para a forma fraca ele conclui naquele artigo que havia fortes evidências empíricas em favor desta hipótese.

Em 1991 Fama fez uma segunda revisão sobre a HME, \citeonline{Fama1991}. Nesta época já havia ficado claro que a distinção entre as formas fraca e semi-forte de eficiência dos mercados eram redundantes. O modelo de passeio aleatório para os preços também havia sido fortemente atacado e parecia não mais se sustentar, \citeonline{Lo1988}.

Um grande número de estudos na literatura financeira confirmou que
os retornos das ações em diferentes horizontes (dias, semanas e meses) podem ser previstos em algum grau por meio de taxas de juros, retorno de dividendos e uma variedade de variáveis macroeconômicas e exibiam claras variações durante o ciclo de econômico. Vários estudos também mostraram que os retornos tendem a ser mais previsíveis quanto mais longo o horizonte de previsão e menor a frequência dos retornos utilizados.

Como o próprio Fama em sua segunda revisão da HME em 1991 notou, um teste para a hipótese de mercado eficiente necessariamente envolve um teste conjunto. Eficiência de mercado e o modelo utilizado para precificação dos ativos. A rejeição da hipótese nula de mercados eficientes pode se dar portanto, em função de o mercado realmente não ser eficiente ou o modelo de precificação utilizado não estar corretamente especificado. Fama conclui que "portanto, eficiência de mercado \emph{per se} não é testável".

No centro da HME estão as três premissas básicas a seguir:

\begin{enumerate}
	\item Racionalidade do investidor: supõe-se que os investidores são racionais, no sentido de que atualizam corretamente suas crenças quando novas informações estão disponíveis. Mesma base das expectativas racionais.
	\item Arbitragem: as decisões de investimento individuais satisfazem a condição de arbitragem, não devem existir lucros potenciais que não tenham sido explorados.
	\item Racionalidade coletiva: os erros aleatórios dos investidores se anulam no mercado. Isso requer que erros individuais devem ser transversalmente independente ou pelo menos apenas fracamente correlacionados.
\end{enumerate}

A hipótese das expectativas racionais é bastante forte, e é improvável
que se mantenha em todos os momentos em todos os mercados. Mesmo que se assuma que no mercado financeiro o processo de aprendizagem e descoberto do preço "correto" ocorra razoavelmente rápido, ainda haverá períodos de turbulência onde os participantes do mercado estarão procurando no escuro, tentando e
experimentando diferentes modelos para os retornos em excesso aquele livre de risco e frequentemente com desvios marcantes
dos resultados racionais comuns. O comportamento de manada e a alta correlação entre os participantes do mercado também podem
levar a desvios da solução de equilíbrio sob expectativas racionais.

Fricções de mercado (i.e. custos de transação ou aquisição de informações) provém outras fontes de previsibilidade do mercado, e portanto, de ineficiências aparentes destes. Estas fricções introduzem um distanciamento entre a correta previsão dos preços e a média de sua estimativa feita pelos agentes de mercado. Em outras palavras, mesmo que os agentes formem suas expectativas racionalmente, estes são impedidos pelas fricções de atuarem no mercado até a completa exaustão da possibilidade de arbitragem, e portanto, os preços correntes refletem as informações disponíveis somente até o limite onde existe lucro a ser obtido descontados os custos de transação e aquisição das informações. Isto faz com que os preços futuros possam ser previstos, muito embora não possam ser economicamente explorados, e o mercado não é eficiente.
