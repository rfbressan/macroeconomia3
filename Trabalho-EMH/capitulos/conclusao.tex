\chapter*{Conclusões}\label{cap:conclusao}
\addcontentsline{toc}{chapter}{Conclusão}

Nos mercados financeiros, a HME é respeitada, mas não é cultuada. É reconhecido que os mercados provavelmente serão eficientes na maior parte do tempo, mas não o tempo todo. Ineficiências podem surgir particularmente durante períodos de importantes mudanças institucionais e tecnológicas. Não é possível saber com antecedência quando e onde as ineficiências do mercado surgem - mas acredita-se que elas surgirão de tempos em tempos. Os operadores do mercado adoram a volatilidade, pois sinalizam notícias e mudanças e são acompanhadas de possibilidades de lucro a serem exploradas. A identificação da previsibilidade explorável tende a ser totalmente diversificada nos mercados de títulos, ações e câmbio. Desalinhamentos entre mercados para diferentes ativos e em diferentes países geralmente apresentam as oportunidades mais importantes. Exemplos incluem arbitragem estatística e macro-arbitragem globais.

Da parte comportamentalista, podemos ver que os pesquisadores não buscam refutar a hipótese dos mercados eficientes, mas sim de poder complementar o viés psicológico dentro dos modelos de alocação de recursos. Em algumas situações em específico, o comportamento humano se torna irracional, o que faz com que os mercados não se ajustem ao valor fundamental de forma adequada. Assim, as finanças comportamentais entram na análise de mercados eficientes como forma de agregar mais complexidade e fatores de impacto no que se trata de alocação de recursos.

A recente literatura financeira parece produzir muitas anomalias de retorno a longo prazo. No entanto, as evidências não sugerem que a eficiência do mercado deva ser abandonada. Consistente com a hipótese de eficiência de mercado de que as anomalias são resultados aleatórios, a aparente reação exagerada dos preços das ações à informação é tão comum quanto a sub-reação. E a continuação de retornos anormais pré-evento é tão frequente quanto a reversão pós-evento.

A maior parte da evidência sugere que qualquer estratégia de investimento supostamente superior deve ser vista com um tanto de ceticismo. O mercado é competitivo o suficiente para que apenas informações ou percepções ligeiramente superiores ganhem dinheiro. As escolhas fáceis foram escolhidas. No final, é provável que a margem de superioridade que qualquer administrador profissional pode adicionar seja tão pequena que apenas o teste estatístico não seja capaz de detectá-la.
